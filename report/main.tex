\documentclass{article}
\usepackage[letterpaper,top=2cm,bottom=2cm,left=3cm,right=3cm,marginparwidth=1.75cm]{geometry}
\usepackage{amsmath}
\usepackage{graphicx}
\usepackage{fancyhdr}
\renewcommand{\footrulewidth}{1pt}
\fancyfoot[L]{\leftmark}
\fancyfoot[C]{\textbf{n° \thepage}} 
\fancyfoot[R]{15.11.2024}
\usepackage{colortbl}
\usepackage{multirow}
\usepackage{booktabs}
\usepackage{hhline}
\usepackage{amsmath}
\usepackage{amsfonts}
\usepackage{graphicx}
\usepackage{subcaption}
\usepackage{float}
\usepackage{array}
\usepackage{listings}
% \usepackage{subcaption}
\usepackage{appendix}
\usepackage{pgfplotstable}
\pgfplotsset{compat=1.18}


\usepackage{siunitx}

\usepackage{booktabs} % For better table formatting

\usepackage{amsthm}
\usepackage{amssymb}
\usepackage{enumitem}


%\usepackage[hidelinks]{hyperref}
\usepackage{hyperref}


\usepackage{cleveref}

\usepackage{multirow}
\usepackage{tabularx}

%Definitions
%\theoremstyle{definition}
\newtheorem{df}{Definition}[section]
\newtheorem{ex}{Example}[section]
%Theorems and Lemmas
\newtheorem{thm}{Theorem}[section]
\newtheorem{corollary}{Corollary}[thm]
\newtheorem{lemma}[thm]{Lemma}
\newtheorem{prop}[thm]{Proposition}

% Adjust spacing above and below displayed equations
\usepackage{setspace}
\setdisplayskipstretch{1.3}
\setlength{\jot}{8pt} 


% Customize lstlistins captions
\newcommand{\codeexcerptname}{Fragment}
\usepackage{caption}
\DeclareCaptionFormat{mylistings}{#1#2#3}
% \captionsetup[lstlisting]{format=mylistings, labelsep=space} % Remove colon
\renewcommand\lstlistingname{\codeexcerptname} % Custom name for lstlistings

% Rename Cref link to code fragments
\Crefname{listing}{\codeexcerptname}{\codeexcerptname s}

% Usage: \codeListing{FIILE.py}{FIRSTLINE}{LASTLINE}{CAPTION}{LABEL}
\newcommand{\codeListing}[5]{%
    \lstinputlisting[
      language=Python,
      caption={\href{https://gitlab.epfl.ch/jfox/optimization-on-manifolds-project-1/-/blame/main/code/#1?ref_type=heads\#L#2}{(\texttt{\detokenize{#1}})}: #4},
      firstline=#2,
      lastline=#3,
      firstnumber=#2,
      label={#5}]{code/#1}%
}


% Macros
\newcommand{\argwrap}[1]{\left(#1\right)}
\newcommand{\argwrapsquare}[1]{\left[#1\right]}
\newcommand{\intS}[1]{\ensuremath{\int_{\Omega}#1 \, dS}}
\newcommand{\intSlong}[1]{\intS{\argwrapsquare{#1}}}
\newcommand{\darg}[2]{\ensuremath{\, \partial_{#2}#1} \, }
\newcommand{\dt}[1]{\ensuremath{\darg{#1}{t}}}
\newcommand{\dS}[1]{\ensuremath{\darg{#1}{S}}}
\newcommand{\dSS}[1]{\ensuremath{\darg{#1}{SS}}}
\newcommand{\dtu}{\dt{u}}
\newcommand{\dSu}{\dS{u}}
\newcommand{\dSSu}{\dSS{u}}
\newcommand{\dtv}{\dt{v}}
\newcommand{\dSv}{\dS{v}}
\newcommand{\dSSv}{\dSS{v}}
\newcommand{\sigmafrac}{\ensuremath{\frac{\sigma^2}{2}}}
\newcommand{\czero}{\ensuremath{C^0((0,T];L^2(\Omega))}}
\newcommand{\seminorm}[1]{\ensuremath{|#1|_V}}
\newcommand{\norm}[1]{\ensuremath{\|#1\|_{L^2(\Omega)}}}
\newcommand{\seminormsq}[1]{\ensuremath{|#1|_V^2}}
\newcommand{\normsq}[1]{\ensuremath{\|#1\|_{L^2(\Omega)}^2}}
\newcommand{\aform}[2]{\ensuremath{\sigmafrac \intS{S^2 \dS{#2} \dS{#1}} + (\sigma^2 - r) \intS{S #2 \dS{#1}} + \intS{r  #1  #2}}}
\newcommand{\auv}{\aform{u}{v}}









\begin{document}

\thispagestyle{empty}

\begin{figure}
\centering
\includegraphics[width=0.5\textwidth]{code/images/LOGO.png}
\end{figure}
\vspace{0.5cm}
\begin{center}
\textsc{ \Large MATH-451: Numerical Approximation of PDEs}
\vspace{1.5cm}
\hrule
\vspace{0.5cm}
{\huge \bfseries Black-Scholes Equation: European Put}
\vspace{0.5cm}
\hrule
\vspace{1.5cm}


\emph{\Large \centering Author:}\\
\vspace{0.3cm}

\large Mattia \textsc{\large Barbiere} (387974)\\

~

\vspace{0.5cm}


\emph{\Large \centering Professor:}\\
\vspace{0.3cm}
\large  Annalisa  \textsc{\large Buffa}\\
~

\vspace{0.5cm}

\large Spring Semester - 2025

\end{center}

\clearpage
\pagenumbering{arabic}

Let us define the following quantities
\begin{df}[Black \& Scholes Equation for European Put Option]\label{def:problem}
Consider the Black \& Scholes equation for the value \( u(S, t) \) of an European Put option:
\[
\begin{cases}
\partial_t u - \frac{\sigma^2}{2} S^2 \partial_{SS} u - r S \partial_S u + r u = 0 & \text{in } \Omega \times (0, T], \\
u(S, 0) = u_0(S) = \max \{ K - S, 0 \},
\end{cases}
\]
where \(\Omega = (S_{\min}, S_{\max})\), \(\sigma\) and \(r\) are strictly positive and bounded constants, together with the following boundary conditions:
\[
\partial_S u(S_{\min}, t) = u(S_{\max}, t) = 0.
\]
\end{df}

\begin{df}[Weighted Sobolev Space]
We introduce the weighted Sobolev space \( V \):
\[
V = \left\{ v : v \in L^2(\Omega), S \frac{\partial v}{\partial S} \in L^2(\Omega), v(S_{\max}) = 0 \right\}.
\]
Endowed with the inner product and norm:
\[
(v, w)_V = \int_\Omega \left( v(S) w(S) + S^2 \frac{\partial v}{\partial S}(S) \frac{\partial w}{\partial S}(S) \right) dS, \quad \| v \|_V = (v, v)_V^{1/2}.
\]
The seminorm
\[
|v|_V^2 = \int_\Omega \left( S \frac{\partial v}{\partial S} \right)^2 dS
\]
is a norm in the Hilbert space \( V \) and we have the following Poincaré inequality:
\[
\| v \|_{L^2(\Omega)} \leq 2 |v|_V, \quad \forall v \in V.
\]
\end{df}

\section{Question 1}
\begin{df}\label{def:a}
    For $u \in \czero$ and $v \in V$ we define the following bilinear form
    \begin{equation*}
        a(u,v) = \auv
    \end{equation*}
\end{df}
\begin{prop}
    The variational form of the problem defined in \Cref{def:problem} is
    \begin{align*}
    &\left( \dtu, v\right) + a(u,v) = \\
    &= \intS{v \dtu} + \auv = 0
    \end{align*}
\end{prop}
\begin{proof}
    Take $v \in V$. Starting from \Cref{def:problem} we multiply by the test function $v$ and integrate over $\Omega$,
    \begin{equation*}
        \intS{v \dtu} - \intS{\sigmafrac S^2  v  \dSSu } - \intS{r S v \dSu} + \intS{r  u  v} = 0.
    \end{equation*}
    We now apply \Cref{thm:by_parts} to the second term. Thus we are left with
    \begin{equation*}
        \intS{v \dtu} - \sigmafrac \left[ \left. \left[S^2 v \dSu \right] \right|_{\partial\Omega} - \intS{\dS{(S^2v)} \dSu}\right] - \intS{r S v \dSu} + \intS{r  u  v} = 0.
    \end{equation*}
    On the boundary we have that
    \begin{align*}
        (S_{min})^2 v(S_{min}) \dSu(S_{min}, t) &= 0 \text{ because of the boundary condition } \dSu(S_{min}, t) = 0\\
        (S_{max})^2 v(S_{max}) \dSu(S_{max}, t) &= 0 \text{ because of the boundary condition } v(S_{max}) = 0
    \end{align*}
    allowing us to eliminate the boundary term. Next using the product rule of the differential gives 
    \begin{equation*}
        \intS{v \dtu} + \sigmafrac \intSlong{2Sv \dSu + S^2 \dSv \dSu} - \intS{r S v \dSu} + \intS{r  u  v} = 0.
    \end{equation*}
    Next using the inner product $\left( \dtu, v\right) = \intS{v \dtu}$ and \Cref{def:a} we can rewrite this as
    \begin{align*}
    &\left( \dtu, v\right) + a(u,v) = \\
    &= \intS{v \dtu} + \auv = 0 \qedhere
    \end{align*}
    \end{proof}

\begin{prop}
    For any $t \in (0,T]$ and any $v \in V$ we have that
    \begin{equation*}
        a(v,v) \geq \frac{\sigma^2}{4} \seminorm{v} - \alpha \normsq{v}
    \end{equation*}
    where $\alpha > 0$ is defined as
    \begin{equation*}
        \alpha = \begin{cases}
            \frac{(\sigma^2 - r)^2}{ \sigma^2}, &\text{ if } \quad r \neq \sigma^2\\
            \varepsilon, &\text{ if } \quad r = \sigma^2\\
        \end{cases}
    \end{equation*}
    with $\varepsilon$ being any positive real number. In particular, $\varepsilon >0$ can be chosen arbitrarily small.
    \begin{proof}
        Recalling \Cref{def:a} we have that
        \begin{align*}
            a(v,v) &= \aform{v}{v}.
        \end{align*}
        By definition of $\normsq{\cdot}$ and $\seminormsq{\cdot}$ we have that
        \begin{equation*}
            a(v,v) = \sigmafrac \seminormsq{v} + (\sigma^2 - r) \intS{S v \dSv} + r \normsq{v}.
        \end{equation*}
        Using properties of the absolute value we can bound this by below
        \begin{equation}\label{eq:first_lower_bound_a}
            a(v,v) \geq \sigmafrac \seminormsq{v} - \left| (\sigma^2 - r) \intS{S v \dSv}\right| + r \normsq{v}.
        \end{equation}
        Let us focus on bounding the second term
        \begin{align*}
            - \left| (\sigma^2 - r) \intS{S v \dSv}\right| &\geq - \left| (\sigma^2 - r) \right| \intS{\left|S v \dSv \right|} && \text{ by the triangle inequality}\\
            &\geq -\left| (\sigma^2 - r) \right| \norm{S \dSv} \norm{v} && \text{ by Cauchy–Schwarz inequality}\\
            &\geq -\left| (\sigma^2 - r) \right| \left( \frac{ \normsq{S \dSv}}{2 \delta} + \frac{\delta \normsq{v}}{2}\right) && \text{ by Young's inequality, $\delta >0$}\\
            & \geq - \frac{\sigma^2}{4} \seminorm{v} - \frac{(\sigma^2 - r)^2 \normsq{v}}{ \sigma^2}
        \end{align*}
        where the last inequality came from choosing $\delta = 2| (\sigma^2 - r) | / \sigma^2$. Plugging this back into \Cref{eq:first_lower_bound_a} give us the following lower bound for $a(v,v)$
        \begin{align*}
            a(v,v) &\geq \frac{\sigma^2}{4} \seminorm{v} - \frac{(\sigma^2 - r)^2}{ \sigma^2} \normsq{v} + r \normsq{v}\\
            &\geq \frac{\sigma^2}{4} \seminorm{v} - \frac{(\sigma^2 - r)^2}{ \sigma^2} \normsq{v}.
        \end{align*}
        by noticing that $r \normsq{v} \geq 0$.
        
        Let us now consider two cases. Firstly, if $r \neq \sigma^2$ we can choose $\alpha = (\sigma^2 - r)^2 / \sigma^2 > 0$ and get
        \begin{equation*}
            a(v,v) \geq \frac{\sigma^2}{4} \seminorm{v} - \frac{(\sigma^2 - r)^2}{ \sigma^2} \normsq{v}
        \end{equation*}
        which is the desired inequality.

        Secondly if $r = \sigma^2$ we can choose any $\alpha >0$ and in particular we can choose $\alpha$ arbitrarily close to $0$. For whatever choice of $\alpha > 0$ the following holds
        \begin{equation*}
            a(v,v) \geq \frac{\sigma^2}{4} \seminorm{v} - \alpha \normsq{v}
        \end{equation*}
        which is again the requested inequality.
    \end{proof}
\end{prop}

\appendix
\section{Green's formulas}
In this appendix we present some useful formulas.

\begin{thm}[Gauss--Ostrogradsky's formula]\label{thm:div}
Let $\Omega \subset \mathbb{R}^n$ be a bounded set with $\partial\Omega \subset C^1$. For $u \in C^1(\bar{\Omega})$,
\[
\int_\Omega u_{x_i}(x)\,dx = \int_{\partial\Omega} u(y)\nu_i(y)\,dS(y), \qquad i = 1, \cdots, n.
\]
\end{thm}

\begin{thm}[Integration by parts formula] \label{thm:by_parts}
Let $\Omega \subset \mathbb{R}^n$ be a bounded set with $\partial\Omega \subset C^1$. For $u, v \in C^1(\bar{\Omega})$,
\[
\int_\Omega u_{x_i}(x)v(x)\,dx = -\int_\Omega u(x)v_{x_i}(x)\,dx + \int_{\partial\Omega} u(y)v(y)\nu_i(y)\,dS(y), \qquad i = 1, \cdots, n.
\]
or, in vectorial form,
\[
\int_\Omega \nabla u(x)v(x)\,dx = -\int_\Omega u(x)\nabla v(x)\,dx + \int_{\partial\Omega} u(y)v(y)\nu(y)\,dS(y)
\]
\end{thm}

\begin{thm}[Green’s identities] \label{thm:green_id}
Let $\Omega \subset \mathbb{R}^n$ be a bounded set with $\partial\Omega \subset C^1$. For $u, v \in C^2(\bar{\Omega})$,
\[
\int_\Omega \Delta u\,dx = \int_{\partial\Omega} \partial_\nu u\,dS,
\]
with $\partial_\nu u = \nabla u \cdot \nu$ being the normal derivative of $u$, and
\[
\int_\Omega v \Delta u\,dx = -\int_\Omega \nabla u \cdot \nabla v\,dx + \int_{\partial\Omega} v\partial_\nu u\,dS = \int_\Omega u\Delta v\,dx + \int_{\partial\Omega} (v \partial_\nu u - u \partial_\nu v)\,dS.
\]
\end{thm}

All previous identities are valid also if the boundary $\partial\Omega$ is only Lipschitz continuous (since Lipschitz functions are differentiable everywhere but a set of points of Lebesgue measure zero).


\end{document}